\documentclass[letterpaper, 12pt]{article}

% packages
\usepackage[utf8]{inputenc}
\usepackage[empty]{fullpage}
\usepackage[hidelinks]{hyperref}
\usepackage{titlesec}
\usepackage{fancyhdr}
\usepackage{enumitem}
\usepackage{tabularx}
\usepackage{siunitx}
\input{glyphtounicode}

% layout stuff
\date{\today}
\pagestyle{fancy}
\fancyhf{}
\fancyfoot{}
\renewcommand{\headrulewidth}{0pt}
\renewcommand{\footrulewidth}{0pt}
\addtolength{\oddsidemargin}{-0.5in}
\addtolength{\evensidemargin}{-0.5in}
\addtolength{\textwidth}{1in}
\addtolength{\topmargin}{-.5in}
\addtolength{\textheight}{1.0in}
\pdfgentounicode=1

% macro definitions
\newcommand{\formula}[2]{
    \item
        \begin{tabularx}{.96\textwidth} { 
                p{\dimexpr.35\linewidth-2\tabcolsep-1.3333\arrayrulewidth}
                p{\dimexpr.65\linewidth-2\tabcolsep-1.3333\arrayrulewidth}
            }
            #1 & #2 \\
            % \hline
        \end{tabularx}
}
\newcommand{\formulaListStart}{\begin{itemize}[leftmargin=0.15in, label={}]}
\newcommand{\formulaListEnd}{\end{itemize}\vspace{-5pt}}

\begin{document}
%%%%%%%%%%%%%%%%%%%%%%%%% HEADER %%%%%%%%%%%%%%%%%%%%%%%%% 
\begin{center}
    {\LARGE \textbf{Quantum Physics Formulas}} \\
    \vspace{5pt}
    \small \href{mailto:davidwrenner@wustl.edu}{David Wrenner} 
    $|$ 
    \today
\end{center}

%%%%%%%%%%%%%%%%%%%%%%%%% FORMULAS %%%%%%%%%%%%%%%%%%%%%%%%% 
\subsubsection*{Constants}
\formulaListStart
    \formula{$c = \SI{3.0e8}{m/s}$}{Speed of light.}
    \formula{$h = \SI{6.626e-34}{J\cdot s}$}{Planck's constant.}
    \formula{$\hbar = \frac{h}{2\pi} = \SI{1.055e-34}{J \cdot s}$}{Planck's constant.}
    \formula{$k_B = \SI{1.381e-23}{J/K}$}{Boltzmann constant.}
    \formula{$e = \SI{1.602e-19}{C}$}{Fundamental charge. Ratio between $J$ and $eV$.}
    \formula{$m_e = \SI{9.109e-31}{kg}$}{Mass of electron.}
    \formula{$m_p = \SI{1.673e-27}{kg}$}{Mass of proton.}
\formulaListEnd

\subsubsection*{Blackbody Radiation}
\formulaListStart
    \formula{$\lambda_{max}T = \SI{2.898e-3}{m \cdot K}$}{Wien's Law.}
    \formula{$I = \sigma T^4$}{Stefan-Boltzmann Law. $ \sigma = \SI{5.67e-8}{W/m^2\cdot K^4}$.}
    \formula{$\frac{dU(f)}{df} = k_B T\times \frac{8\pi V}{c^3} f^2$}{Spectral energy density (classical theory).}
    \formula{$\frac{dU(f)}{df} = \frac{hf}{e^{hf / k_BT}-1}\times \frac{8\pi V}{c^3} f^2$}{Spectral energy density (Planck).}
\formulaListEnd

\subsubsection*{The Photoelectric Effect}
\formulaListStart
\formula{$E = hf = \frac{hc}{\lambda}$}{Energy of a photon (Einstein).}
\formula{$KE_{max} = hf - \phi$}{Max KE of an electron ejected by a photon.}
\formulaListEnd

\subsubsection*{The Compton Effect}
\formulaListStart
\formula{$E = pc$}{Energy of an object with zero mass: energy of a photon}
\formula{$p = \frac{h}{\lambda}$}{Momentum of a photon.}
\formula{$\lambda ' - \lambda = \frac{h}{m_ec}(1-cos(\theta))$}{Compton shift equation. $\lambda ' \geq \lambda, E' \leq E.$}
\formulaListEnd

\subsubsection*{Particle vs. Wave Behavior}
\formulaListStart
\formula{$\lambda << D$}{Characteristic of particle behavior. $D$ is relevant dimension of the experiment.}
\formula{$\lambda \geq D$}{Characteristic of wave behavior.}
\formulaListEnd
\newpage

\subsubsection*{Matter Waves}
\formulaListStart
\formula{$2d\cdot sin(\theta) = m\lambda$}{Bragg Law. $m\in \mathbf{Z}^+$}
\formula{$\lambda=\frac{h}{p}$}{de Broglie wavelength.}
\formula{$f=\frac{E}{h}$}{Frequency of a matter wave.}
\formula{$v_{wave} = f\lambda = \frac{E}{p}$}{Velocity of a matter wave.}
\formula{$k\equiv\frac{2\pi}{\lambda}$}{Wave number definition.}
\formula{$\omega\equiv\frac{2\pi}{T}$}{Angular frequency definition.}
\formula{$p=\frac{h}{\lambda}=\hbar k$}{Fundamental wave-particle relationship.}
\formula{$E=hf=\hbar\omega$}{Fundamental wave-particle relationship.}
\formula{$-\frac{\hbar^2}{2m} \frac{\partial^2}{\partial x^2}\Psi(x,t) = i \hbar \frac{\partial}{\partial t}\Psi(x,t)$}{Schrodinger Equation for free particles, without external forces.}
\formula{$\lambda = \frac{h}{\sqrt{3m k_B T}}$}{Wavelength of a thermal particle.}
\formulaListEnd

\subsection*{The Plane Wave}
\formulaListStart
\formula{$\Psi(x,t) = Ae^{i(kx-\omega t)}$}{SE solution.}
\formula{$\frac{\hbar^2k^2}{2m}=\hbar\omega$}{Result from evaluating SE with solution.}
\formula{$\frac{p^2}{2m}=E=\frac{1}{2}mv^2$}{By wave-particle relationships.}
\formulaListEnd

\subsection*{The Uncertainty Principle}
\formulaListStart
\formula{$\Delta Q = \sqrt{\frac{\sum_i (Q_i - \bar{Q})^2 n_i}{\sum_i n_i}}$}{Definition of uncertainty (standard deviation).}
\formula{$\Delta p_x \propto \frac{1}{\Delta x}$}{Single slit experiment: width of diffraction pattern inversely proportional to slit width.}
\formula{$\Delta p_x \Delta x \geq \frac{\hbar}{2}$}{Heisenberg uncertainty principle.}
\formula{$\Delta p_x \Delta x = \frac{\hbar}{2}$}{Special case: Gaussian. $\Psi(x,0) = Ce^{-(\frac{x}{2\epsilon})^2}$. $\Delta x \propto \epsilon$, $\Delta p_x \propto \frac{1}{\epsilon}$ }
\formula{$\Delta E \Delta t \geq \frac{\hbar}{2}$}{Relation of particle energy and time of decay.}
\formulaListEnd

\newpage
\subsection*{The Bohr Model of the H atom}
\formulaListStart
\formula{$E_{classical} = -\frac{e^2}{8\pi\epsilon_0r}$}{Energy of orbiting electron (classical theory).}
\formula{$\L=n\hbar$}{Bohr postulate: angular momentum takes on only discrete values. $n\in \mathbf{Z^+}$}.
\formula{$v=\frac{n\hbar}{mr}$}{}
\formula{$r=a_0n^2=\frac{(4\pi\epsilon_0)\hbar^2}{me^2}n^2$}{Bohr radii for H atom. $a_0=\SI{.0529}{nm}$}.
\formula{$E = \frac{-me^4}{32\pi^2\epsilon_0^2\hbar^2}\frac{1}{n^2}
= \SI{-13.6}{eV}\frac{1}{n^2}$}{Bohr energy for H atom.}
\formulaListEnd

\subsection*{Math Basis for Uncertainty}
\formulaListStart
\formula{$\Psi(x) = \frac{1}{2\pi}
\int_{-\infty}^{\infty} A(k)e^{ikx} \,dx
$}{$\Psi(x)$ as a sum of plane waves of amplitude $A(k)$.}
\formula{$A(k) = \frac{1}{2\pi}
\int_{-\infty}^{\infty} \Psi(x)e^{-ikx} \,dx
$}{Fourier transform of $\Psi(x)$.}
\formulaListEnd

\end{document}
